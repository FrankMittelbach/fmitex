\documentclass{ltugboat}


\usepackage{amssymb}

\newcommand\LPack[1]{\texttt{#1}}
\newcommand\Prog[1]{\texttt{#1}}
\newcommand\pdfTeX{pdf\TeX}

\newcommand\tofu{$\square$}

\setcounter{secnumdepth}{1}


\author{Frank Mittelbach}
\title{Preventing Tofu with \pdfTeX{} and Unicode engines}
\address{Mainz, Germany}
\netaddress{https://www.latex-project.org}
\personalURL{https://ctan.org/pkg/unicodefonttable}

\begin{document}

\maketitle

\begin{abstract}

\end{abstract}

\tableofcontents

\section{With tofu through the years}

Tofu is not just an essential ingredient for many Asian dishes, it is
also the nick name for little squares that are produced by many
bowsers when they get asked to render a character for which they do
not have a glyph available.

Especially in the early days of the World Wide Web,
websites in foreign languages (from the perspective of your computer)
got often littered with such squares, making text comprehension quite
difficult if not impossible in some cases. So instead of getting
\begin{quote}
  ?`But aren't Kafka's Schlo{\ss} \&
  {\AE}sop's {\OE}uvres often na{\"\i}ve vis-\`{a}-vis the d{\ae}monic
  \textit{ph{\oe}nix's official r\^{o}le} in fluffy souffl\'{e}s?
\end{quote}
you might have seen something like
\begin{quote}
  \tofu But aren't Kafka's Schlo{\tofu} \tofu{} {\tofu}sop's
  \tofu{}uvres often na{\tofu}ve vis-\tofu{}-vis the d{\tofu}monic
  \textit{ph{\tofu}nix's official r\tofu{}le} in fluffy
  souffl\tofu{}s?
\end{quote}
Over the years the situation with browsers improved (partly because
using inferior fonts deemed acceptible as long as they could render
the needed glyphs), but even nowadays you may find tofu-littered sites
when you direct your browser to it or perhaps worse those where you
browser thinks it can show you the glyphs but render the wrong ones.

While with browsers you may accept a certain imperfection in the
rendering, tofu in printed material is quite unacceptable and
typesetting systems should always use the correct glyphs or at least
tell you very explicitly if for some reason they are unable to do so,
to allow you to can apply some corrective actions. In the remainder of
this article we will discuss how \TeX{} and in particular \LaTeX{} is
doing in this respect and what a user can or has to do to avoid some
capital blunder.

%\subsection{The first \TeX{} years}
\subsection{Early  vegitarian dishes with \TeX{}}

In the early days of \TeX{} the use of fonts was easy because you
could use any font you wanted as long as it was called Computer
Modern.

In other words there was essentially only one set of fonts available
for use with \TeX\ and the glyphs it contained and how to address them
was described in the \TeX{}book~\cite{}. Furthermore, all fonts only
contained 128 glyphs, i.e., essentially the base Latin characters, a
few accents to construct diacritical characters using the \cs{accent}
primitive and a few other symbols such as \textdagger, \$ and so forth
to be accessed through command names.

Thus, once you learned the construction methods and memorized the
control sequences for accessing the existing symbols you could be sure
that the characters you used would faithfully appear in the printed
result. Of course, part of the reason was that the glyph set was
fairly limited and already for any Latin-based language other than
English posed series issues, be it that necessary glyphs where missing
or only available as constructed characters (whenever accents where
involved) which prevented \TeX{} from applying hyphenation.

So when \TeX{} got more and more popular outside the English-speaking
world there was considerable pressure on Don Knuth (largely by
European user groups) to extend \TeX{} so that it could better handle
languages with larger character sets. At the 1989 \TeX{} conference in
Stanford we finally managed to convince Don to reopen (in a limited
way) \TeX{} development and to produce \TeX~3. This version of \TeX{}
was then able to deal with more than one language within a document
(use multiple hyphenation patterns) and support 8-bit input and output
(that is 256 characters in a font).

While this enabled the use of different input code pages, as common in
those days in computers depending on the country and solved the
problem of hyphenating words containing accented characters (by using
fonts with precomposed glyphs), it also posed new challenges.

Depending on the code page used in the computer when writing a
document the same keyboard character might got associated with a
different number (between 0 and 255) and that number had to be mapped
to the right slot in a font to produce the glyph that was originally
intended. So the days of input number equals font glyph position was
definitely over and the \TeX{} world had to come up with a more
elaborate scheme to translate one into the other to avoid missing or
wrong characters in the output.


\subsection{The \LaTeXe{} solution}

For \LaTeX{} the solution came in form of the New Font Selection
Scheme (NFSS)~\cite{} and in particular with the packages
\LPack{inputenc} (for managing input in different code pages and
mapping it to a standard internal representation) and \LPack{fontenc}
(for translating this internal representation to the correct glyph
positions in different fonts).


\subsubsection{Introducing font encodings}

\LaTeX{} classified the font encodings and gave them names such as
\texttt{OT1}, \texttt{T1}, \texttt{T2A}, \texttt{T2B}, etc. Each such
font encoding defined which glyphs are in a font using that encoding
and where each glyph was stored in the font. Thus, if you had two
different fonts with the same encoding you could exchange one for the
other and still be one hundred percent sure\footnote{Well, more 99\%
  because sometimes fonts claimed to be in one encoding but didn't
  faithfully follow its specification, e.g., didn't provide all
  glyphs or sometimes even placed wrong glyphs into some slots.}
that your document would still typeset correctly without any missing
or incorrect glyphs in the output.

In practice only a small number of font encodings ever got used and
new fonts usually were made available in these ``popular'' encodings
by providing the necessary font re-encodings through the virtual font
mechanism or through re-encodings done by device drivers such as
\Prog{dvips} or directly in the engine (in case of \Prog{pdftex}).

As overall result, life for \LaTeX{} users got again fairly easy after
1994 and remained this way well into this century as just by
specifying which font encoding to use documents normally would typeset
without any defects regardless of the font family that got used. And
due to the fact that users writing in Latin-based languages
essentially every font available was available in \texttt{T1} encoding
it was also clear which glyphs were available and those were than
available universally.

\subsubsection{Pitfalls with missing input encodings}

Of course, there was also the need to specify the input encoding---at
least if one wanted to input accented characters directly from the
keyboard instead of using constructs like \verb=\"a=. One problem in
this respect was that, depending on the language you were writing in,
it sometimes worked even without specifying the input encoding. This
was possible as the \texttt{T1} font encoding was nearly identical to
the quite common \texttt{latin1} input encoding.\footnote{For example,
  with French texts it worked throughout. However, with German only
  the ``umlauts'' worked, but the sharp s ``ß'' generated a different
  character.}  Years later, omitting the input encoding declaration
even when it worked initially finally backfired: once \LaTeX{} moved
on to make UTF-8 the default encoding, documents stored in legacy
encodings failed if they didn't contain an input
declaration.\footnote{The remedy for such old documents is then to
  either add the missing declaration or re-encode the old source and
  store them in UTF-8.}


\section{Unicode}

One of the goals of Unicode was to uniquely identify each and every
glyph and thereby avoiding some of the possible translation problems
that occurred because a text was written under the assumption of one
(8-bit) encoding but interpreted later on under a different encoding.

\vfill

\makesignature

\end{document}

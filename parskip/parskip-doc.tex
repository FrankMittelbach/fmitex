\documentclass[pagesize=auto, fontsize=14pt, DIV=10, parskip=half]{scrartcl}

\usepackage{lmodern}
\usepackage[T1]{fontenc}
\usepackage{textcomp}
\usepackage{booktabs}
\usepackage{microtype}
\usepackage{hyperref}

\newcommand*{\mail}[1]{\href{mailto:#1}{\texttt{#1}}}
\newcommand*{\pkg}[1]{\textsf{#1}}
\newcommand*{\cls}[1]{\textsf{#1}}
\newcommand*{\cs}[1]{\texttt{\textbackslash#1}}
\makeatletter
\newcommand*{\cmd}[1]{\cs{\expandafter\@gobble\string#1}}
\makeatother

\addtokomafont{title}{\rmfamily}

\title{The \pkg{parskip} package}
\author{H. Partl\and Robin Fairbairns \and Frank Mittelbach}
\date{2018/08/24}


\begin{document}

\maketitle

This is \texttt{parskip.sty} by H.~Partl, TU Wien, as of 19 Jan 1989.
Addition (originally from Donald Arseneau) added 2001-12-13 by Robin
Fairbairns. A reimplementation (adding options) was done 2018-08 by Frank Mittelbach.

The package can be used with any document class at any size.
By default it produces the following Paragraph Layout:

\begin{quote}
  Zero Parindent and non-zero Parskip. The stretchable glue in \cmd{\parskip}
  helps \LaTeX\ in finding the best place for page breaks.
\end{quote}

In addition, the package adjusts the skips between list items, TOCs and headings.

With the package option \texttt{skip} it is possible to explicitly
specify the vertical space between paragraphs.

With the package option \texttt{indent} it is possible to explicitly
the paragraph indentation. Using this option without a value keeps the
document class indentation. Without the option the indentation is set
to zero.

With package option \texttt{parfill}, the package also adjusts
\cmd\parfillskip{} to impose a minimum space at the end of
the last line of a paragraph.

This package is no more than quick fix; the `proper' way to achieve
effects as far-reaching as this is to create a new class.  An
example class is to be found in the \cls{ntgclass} set:
\cls{artikel3.cls}

The \cls{koma-script} bundle classes and the \cls{memoir} class all
provide similar functionality, and their respective documentation
files discuss the pros (such as they are) and cons of this approach.

\end{document}

% \iffalse meta-comment
%%
%% File: widows-and-orphans.dtx (C) Copyright 2017-2018 Frank Mittelbach
%
% It may be distributed and/or modified under the conditions of the
% LaTeX Project Public License (LPPL), either version 1.3c of this
% license or (at your option) any later version.  The latest version
% of this license is in the file
%
%    http://www.latex-project.org/lppl.txt
%
%
% The development version of the bundle can be found at
%
%    https://github.com/FrankMittelbach/fmitex/widows-and-orphans
%
% for those people who are interested.
% \fi
% \iffalse
%<*driver>
\documentclass
%    [final,nonumber]
    {l3doc-TUB}

\newcommand\bibname{References}  % why???

\IndexPrologue
  {\section*{Index}
    \markboth{Index}{Index}
    \addcontentsline{toc}{section}{Index}
    The italic numbers denote the pages where the 
    corresponding entry is described, 
    numbers underlined point to the definition, 
    all others indicate the places where it is used.}
\EnableCrossrefs
\CodelineIndex
\begin{document}

\onecolumn         % ltugboat.sty macro for single-column articles
\SelfDocumenting   % ltugboat.sty macro that changes page parameters
\DocInput{widows-and-orphans.dtx}
\makesignature
\end{document}
%</driver>
% \fi
%
% \DoNotIndex{\cs_new:Npn,\cs_new_nopar:Npn,\csname,\endcsname,\fi:,\else:}
% \DoNotIndex{\tl_new:N,\bool_new:N}
% \DoNotIndex{\if_meaning:w,\iftrue,\int_new:N,\int_set:Nn}
% \DoNotIndex{\prop_new:N,\tl_put_left:Nn}
% \DoNotIndex{\bool_do_until:Nn,\bool_gset_false:N,\bool_gset_true:N}
% \DoNotIndex{\bool_if:NTF,\bool_set_false:N,\bool_set_true:N}
% \DoNotIndex{\clist_map_break:n,\clist_map_inline:nn}
% \DoNotIndex{\cs_if_eq:NNTF,\exp_args:Nc}
% \DoNotIndex{\int_case:nnTF,\int_case:nnF,\int_case:nnF,\int_incr:N}
% \DoNotIndex{\int_to_arabic:n,\int_to_roman:n}
% \DoNotIndex{\keys_define:nn,\keys_set:nn}
% \DoNotIndex{\msg_err:nn,\msg_new:nnnn,\msg_redirect_module:nnn}
% \DoNotIndex{\msg_warning:nn,\msg_warning:nnn,\msg_error:nn}
% \DoNotIndex{\prop_clear:N,\prop_get:NnNTF,\prop_put:Nnn,\prop_show:N}
% \DoNotIndex{\prg_new_conditional:Npnn}
% \DoNotIndex{\prg_return_false:,\prg_return_true:}
% \DoNotIndex{\prop_get:NVNT,\prop_put:NVn,\bool_if:NF,\bool_if:NT}
% \DoNotIndex{\marginpar,\space,\ignorespaces,\thepage}
% ^^A\DoNotIndex{\RequirePackage,\NeedsTeXFormat,\NewDocumentCommand}
%
% \title{The \pkg{windows-and-orphans} package}
% \author{Frank Mittelbach}
% \date{Printed \today\thanks{For the package version
%   see the \texttt{\string\ProvidesExplPackage} line in the code.}}
% \address{Mainz, Germany}
% \netaddress{https://www.latex-project.org}
%
% \maketitle
%
% \begin{abstract}
%   The \pkg{windows-and-orphans} package checks page or column breaks
%   for issues with widow or orphans lines and issues warnings if such
%   problems are detected. In addition it also checks and complains
%   about breaks involving hyphenated words and warns about display
%   formulas directly after a page break (if they are allowed by the
%   document parameter settings, which by default isn't the case).
%
%   A general discussion of the problem of widows and orphans and
%    suggestions how to resolve
%   it is given in~\cite{tub-wao}.
% \end{abstract}
%
% \tableofcontents
%
% \newcommand\option[1]{\texttt{#1}}
% \newcommand\ovalue[1]{\texttt{#1}}
%
% \section{Overview}
%
% To determine if a widow or orphan has occurred at a column/page
% break the package analyses the \cs{outputpenalty} that triggered the
% break. As \TeX{} adds special penalties to widow and orphan lines
% (\cs{widowpenalty} and \cs{clubpenalty}), we can hope to identify
% them, provided the penalties have unique values so that we don't end
% up with false positives.
%
% The package therefore analyzes the different parameter values and if
% necessary adjusts the settings slightly so that all possible
% combinations that can appear in documents have unique values and can
% thus be identified.
%
% All that is necessary then, is to hook into the output routine check
% if \cs{outputpenalty} has a value that matches one of the
% problematic cases and if so issue a warning.
%
% Besides widows and orphans it is possible to detect other cases
% controlled through penalty parameters, e.g., \cs{brokenpenalty} that is
% added if a line ends in a hyphen. So by including this parameter
% into the checks, we can identify when that happens at the end of a
% column and issue a warning too.
%
% We also do this for \cs{predisplaypenalty} controlling a break just
% in front of a math display. This is normally set to \texttt{10000}
% so such breaks don't happen in standard \LaTeX{}, but if the value
% is lowered it becomes possible (and is a possible issue to look at).
%
% \subsection{Options}
%
% The package has a number of key/value options to adjust its
% behavior.
%
% The option \option{check} is used to define what happens
% when an issue is found (default is \ovalue{warning}, other
% possibilities are \ovalue{error}, \ovalue{info} and \ovalue{none}).
%
% The options \option{orphans} and \option{widows} can be used to set
% up reasonable parameter values (default is to use whatever the class
% defines). Possible values are \ovalue{prevent}, \ovalue{avoid} or
% \ovalue{default} (the latter means use standard \LaTeX{} defaults).
%
% To set all parameters in one go you can use \option{prevent-all},
% \option{avoid-all} or \option{default-all}. These options also assign
% values to \cs{brokenpenalty} and \cs{predisplaypenalty}.
%
% \subsection{User commands}
%
% The package provides three user-level commands.
%
% \begin{function}{\WaOsetup}
%   \begin{syntax}
%     \cs{WaOsetup} \meta{comma list}
%   \end{syntax}
%   This command accepts any of the package options and allows
%   adjusting the package behavior in mid-document if necessary.
% \end{function}

% \begin{function}{\WaOparameters}
%   \begin{syntax}
%     \cs{WaOparameters}
%   \end{syntax}
%    With this command a listing of the parameter combinations and
%    their values is provided.
% \end{function}

% \begin{function}{\WaOignorenext}
%   \begin{syntax}
%     \cs{WaOignorenext}
%   \end{syntax}
%   This command will direct the package to not generate warnings for
%    the current page or columns (if we know that they can't be corrected).
% \end{function}
%
%
% \StopEventually{
% \begin{thebibliography}{1}
% \bibitem{tub-wao}
% Frank Mittelbach.
% \newblock Managing forlorn paragraph lines (a.k.a.\ widows and orphans)
%  in \LaTeX{}.
% \newblock TUGboat ???
% \end{thebibliography}
% \setlength\IndexMin{200pt}  \PrintIndex  }
%
%
% \section{The Implementation}
%
% The package is implemented in \pkg{expl3} so the first thing we do
% is to define a prefix for our internal commands, so that we can
% write \texttt{@@} in the source when we mean \texttt{__fmwao}
% indicating that something is internal.\footnote{\texttt{l3docstrip}
% will expand that for us so in the \texttt{.sty} file we get the
% longer names with double underscores even though the real source
% just contains \texttt{@@} all over the place. The same is done by
% the \texttt{l3doc} class for the printed documentation you are
% currently reading.}
%
%    \begin{macrocode}
%<@@=fmwao>
%    \end{macrocode}
%
% Then we check that we are running on top of \LaTeXe{} and load the
% two packages we want to use: \pkg{xparse} for the user interface and
% \pkg{l3keys2e} for key/val syntax of options. They do load the
% \pkg{expl3} code so we are ready to roll afterwards.
%
%    \begin{macrocode}
\NeedsTeXFormat{LaTeX2e}
\RequirePackage{xparse,l3keys2e}
%    \end{macrocode}
%
% Then we announce the package to the world at large. This declaration
% will also tell \LaTeX{} that this is an \pkg{expl3} package and that
% from now on the \pkg{expl3} conventions are to be obeyed, e.g.,
% \texttt{_} and \texttt{:} can appear in command names and white
% space is automatically ignored (so no need for \texttt{\%} all over
% the place).
%
%    \begin{macrocode}
\ProvidesExplPackage{widows-and-orphans}{2018/09/01}{v1.0a}
                    {Detecting widows and orphans (FMi)}
%    \end{macrocode}
%
%
%  \begin{macro}[int]{\@makecol}
%    
%    As mentioned in the introduction we want to check the value of
%    \cs{outputpenalty} inside the output routine and for this we need
%    to add some code to the output routine.
%
%    We add that to the front of \cs{@makecol} the macro that
%    assembles the actual column. This way it only gets executed if we
%    make a column or a page but not when the output routine is
%    triggered because of a float or \cs[no-index]{marginpar}.
%    \begin{macrocode}
\tl_put_left:Nn \@makecol { \@@_test_for_widows_etc: }
%    \end{macrocode}
%  \end{macro}
%
%
% \subsection{Checking \cs{outputpenalty}}
%
%
%  \begin{macro}{\g_@@_gen_warn_bool}
%    To be able to suppress checking we define a global boolean
%    variable which by default is set to \texttt{true}.
%    \begin{macrocode}
\bool_new:N \g_@@_gen_warn_bool
\bool_gset_true:N \g_@@_gen_warn_bool
%    \end{macrocode}
%  \end{macro}
%
%
%
%  \begin{macro}{\@@_test_for_widows_etc:}
%
%    What are the different values related to orphans and widows and
%    the like that the can appear in \cs{outputpenalty}? Here is the
%    basic list:
%    \begin{description}
%
%    \item[$\cs{widowpenalty} + \cs{interlinepenalty} \to$] if the
%    break happens on the second last line of a paragraph and the
%    paragraph is not followed by a math display.
%
%    \item[$\cs{displaywidowpenalty} + \cs{interlinepenalty} \to$] if
%    the break happens on the second last line of a paragraph and the
%    paragraph is followed by a math display.
%
%    \item[$\cs{clubpenalty} + \cs{interlinepenalty} \to$] if the
%    break happens after the first line of a paragraph and the
%    paragraph has more than two lines.
%
%    \item[$\cs{clubpenalty}+\cs{widowpenalty} + \cs{interlinepenalty}
%    \to$] if the break happens after the first line of a paragraph in
%    a two line paragraph (thus this line is also the second last
%    line).
%
%    \item[$\cs{clubpenalty}+\cs{displaywidowpenalty}  +
%    \cs{interlinepenalty} \to $] if the
%    break happens after the first line of a paragraph in a two line
%    paragraph and a math display follows.
%
%    \end{description}
%    That's it for widows and orphans. If we also consider hyphenated
%    breaks then we get a further set of cases, namely all of the
%    above with \cs{brokenpenalty} added in and the case of
%    \cs{brokenpenalty} on its own (with just \cs{interlinepenalty} added).
%    
%    \begin{macrocode}
\cs_new:Npn \@@_test_for_widows_etc: {
%    \end{macrocode}
%    So here is the main test. We compare \cs{outputpenalty} with each
%    case until we have a hit or run out of cases. Instead of adding
%    \cs{interlinepenalty} to each case we substract it once from
%    \cs{outputpenalty} as that makes the comparison a little more
%    efficient.
%
%    If we get a hit, we execute the corresponding code which is
%    always either \cs{@@_problem_identified:n} or
%    \cs{@@_problem_identified:nn} (i.e., one or two arguments) to
%    issue the warning. The arguments are to select the right warning
%    text and customize it further if necessary. For example, in the
%    first case it is a ``widow'' problem and the text in the message
%    should start with ``\texttt{Widow}'' while in the second case it
%    is also a ``widow'' problem but the text will say
%    ``\texttt{Display\string~ widow}''.\footnote{If you haven't seen
%    much \pkg{expl3} code you may wonder about the
%    \texttt{\string~}. As the code ignores spaces we have to mark up
%    real spaces and for this the tilde is used.}  This just saves a
%    bit of space when diffeent cases share more or less the same
%    message text.
%    \begin{macrocode}
  \int_case:nnF { \outputpenalty - \interlinepenalty }
    {
      { \widowpenalty }
        { \@@_problem_identified:nn{widow}{Widow} }
      { \displaywidowpenalty }
        { \@@_problem_identified:nn{widow}{Display~ widow} }
      { \clubpenalty }
        { \@@_problem_identified:n{orphan} }
      { \clubpenalty + \widowpenalty }
        { \@@_problem_identified:nn{orphan-widow}{} }
      { \clubpenalty + \displaywidowpenalty }
        { \@@_problem_identified:nn{orphan-widow}{display} }
%    \end{macrocode}
%
%    A similar issue comes from a hyphen at the end of a column or
%    page in which case \TeX{} adds \cs{brokenpenalty} so we can test
%    against that:
%    \begin{macrocode}
      { \brokenpenalty }
        { \@@_problem_identified:n{hyphen} }
%    \end{macrocode}
%    However, I said ``\TeX{} adds'' which means if a widow line also
%    ends up in a hyphen then the penalty will be the sum of both
%    individual penalties. So all the case above need to be repeated
%    with \cs{brokenpenalty} added to the value.
%    We generate the same warnings, though---e.g., we will say ``Widow
%    detected'' and not ``Hyphenated widow line detected'' as the
%    latter seems to be overkill.
%    \begin{macrocode}
      { \brokenpenalty + \widowpenalty }
        { \@@_problem_identified:nn{widow}{Widow} }
      { \brokenpenalty + \displaywidowpenalty }
        { \@@_problem_identified:nn{widow}{Display~ widow} }
      { \brokenpenalty + \clubpenalty }
        { \@@_problem_identified:n{orphan} }
      { \brokenpenalty + \clubpenalty + \widowpenalty }
        { \@@_problem_identified:nn{orphan-widow}{} }
      { \brokenpenalty + \clubpenalty + \displaywidowpenalty }
        { \@@_problem_identified:nn{orphan-widow}{display} }
%    \end{macrocode}
%    Finally there is \cs{predisplaypenalty} that we may check as well
%    (in case it was set to a value lower than \texttt{10000}). If it
%    appears it means we have a display at the very top of the page.
%    We reuse the ``widow'' warning but this time say
%    ``\texttt{Lonely\string~ display}'' in the second argument.
%    Note, that this case does not get \cs{interlinepenalty} added by
%    \TeX{}, so we have to undo the optimization above.
%    \begin{macrocode}
      { \predisplaypenalty - \interlinepenalty }
        { \@@_problem_identified:nn{widow}{Lonely~ display} }
    }
%    \end{macrocode}
%    The last argument of \cs[no-index]{int_case:nnF} is executed in
%    the ``false'' case, i.e., when no match has been found.  In that
%    case we check the status of \cs{g_@@_gen_warn_bool} and
%    if that is also ``false'' (i.e., if we have been asked not to
%    generate warnings) we issue an error message. Why?  Because the
%    user asked us to explicitly, to ignore problems on the current
%    page, but there was nothing wrong here. This either means a
%    problem got corrected or the request was really meant for a
%    different page. Either way it is probably worth checking.
%
%    \begin{macrocode}
    { \bool_if:NF \g_@@_gen_warn_bool
        { \msg_error:nn{widows-and-orphans}{no-problem} } }
%    \end{macrocode}
%    Finally, we make sure that the next page or column is again
%    checked.
%    \begin{macrocode}
  \bool_gset_true:N \g_@@_gen_warn_bool
}
%    \end{macrocode}
%  \end{macro}



%  \begin{macro}{\@@_problem_identified:n}
%  \begin{macro}{\@@_problem_identified:nn}
%    These commands prepare for generating a warning, but only if we
%    are supposed to, i.e., if \cs{g_@@_gen_warn_bool} is true.
%    \begin{macrocode}
\cs_new:Npn \@@_problem_identified:n #1 {
   \bool_if:NT \g_@@_gen_warn_bool
               { \msg_warning:nn{widows-and-orphans}{#1} }
}
\cs_new:Npn \@@_problem_identified:nn #1 #2 {
   \bool_if:NT \g_@@_gen_warn_bool
               { \msg_warning:nnn{widows-and-orphans}{#1}{#2} }
}
%    \end{macrocode}
%  \end{macro}
%  \end{macro}

      
% \subsection{The messages to the user}
%
%
%  \begin{macro}{\@@_this_page:}
%  \begin{macro}{\@@_next_page:}
%    For displaying nice messages to the user we need a few helper
%    commands. The two here show the page number of the current or
%    the next page. They are semi-smart, that is they will recognize
%    if the document uses roman numerals and if so display the number
%    as a roman numeral (but in all other cases is will use arabic).
%    \begin{macrocode}
\cs_new:Npn \@@_this_page: { \@@_some_page:n   \c@page       }
\cs_new:Npn \@@_next_page: { \@@_some_page:n { \c@page + 1 } }  
%    \end{macrocode}
%  \end{macro}
%  \end{macro}


%  \begin{macro}{\@@_some_page:n}
%  \begin{macro}{\@@_roman_thepage:}
%    This macro will first compare \cs[no-index]{thepage} against the code that
%    would be used in case of a roman numeral representation and then
%    displays its argument using either arabic numbers or roman numerals.
%    \begin{macrocode}
\cs_new:Npn \@@_some_page:n #1 {
  \cs_if_eq:NNTF \thepage \@@_roman_thepage:
    { \int_to_roman:n } { \int_to_arabic:n }
  { #1 }
}                               
%    \end{macrocode}
%    \cs{@@_roman_thepage:} just stores the default definition of
%    \cs[no-index]{thepage} if page numbers are represented by roman numerals
%    for use in the comparison above.
%    \begin{macrocode}
\cs_new_nopar:Npn \@@_roman_thepage: {\csname @roman\endcsname \c@page}
%    \end{macrocode}
%  \end{macro}
%  \end{macro}




%  \begin{macro}[TF, int]{\legacy_switch_if:n}
%    To evaluate \LaTeXe{} boolean switches in a nice way, we need a
%    conditional which will probably make it into the \pkg{expl3}
%    code in this or a similar form, but right now it is missing.
%    \begin{macrocode}
\prg_new_conditional:Npnn \legacy_switch_if:n #1 {p, T , F , TF }
  { \exp_args:Nc\if_meaning:w { if#1 } \iftrue \prg_return_true:
                                        \else: \prg_return_false: \fi: }
%    \end{macrocode}
%  \end{macro}
%
%
%   \bigskip
%
%
%    The first message is issued if we have been directed to ignore a
%    problem and there wasn't one:
%    \begin{macrocode}
\msg_new:nnnn {widows-and-orphans} {no-problem}
  { No~ problem~ to~ suppress~ on~ this~ page! }
  { Suppression~ of~ a~ widow~ or~ orphan~ problem~ was~ requested~
    but~ on~ the~ current~ page~ there~ doesn't~ seem~ to~ be~ any.~
    Maybe~ the~ text~ was~ changed~ and~ the~ request~ should~ get~ (re)moved?}
%    \end{macrocode}
%
%
%   \bigskip
%
%
%
%
%    The next message is about orphans. They can appear at the bottom of
%    the first or the second column of the current page (if we are in
%    two-column mode so we check for this and adjust the message accordingly.
%    \begin{macrocode}
\msg_new:nnnn {widows-and-orphans} {orphan}
  { Orphan~ on~ page~ \@@_this_page:
    \legacy_switch_if:nT {@twocolumn}
      { \space ( \legacy_switch_if:nTF {@firstcolumn}
                                       { first~ } { (second~ } column) }
  }
  { Check~ out~ the~ page~ and~ see~ if~ you~ can~ avoid~ the~ orphan.}
%    \end{macrocode}
%
%
%   \bigskip
%
%
%    A hyphen at the end of a page or column requires more or less the
%    same message, so this could have been combined with the previous one.
%    \begin{macrocode}
\msg_new:nnnn {widows-and-orphans} {hyphen}
  { Hyphen~ in~ last~ line~ of~ page~ \@@_this_page:
    \legacy_switch_if:nT {@twocolumn}
      { \space ( \legacy_switch_if:nTF {@firstcolumn}
                                       { first~ } { (second~ } column) } }
  { Check~ out~ the~ page~ and~ see~ if~ you~ can~ get~
    a~ better~ line~ break. }
%    \end{macrocode}
%
%
%   \bigskip
%
%
%    Widows need a different logic since we detect them when cutting a
%    previous page or column but the widow is on the following one.
%    This code works for ``widows'', ``display widows'' as well as
%    math displays by just changing the first word (or words), so here
%    we use an additional argument:
%    \begin{macrocode}
\msg_new:nnnn {widows-and-orphans} {widow}
  { #1~ on~ page~ 
    \legacy_switch_if:nTF {@twocolumn}
      { \legacy_switch_if:nTF {@firstcolumn}
          { \@@_this_page: \space (second~ }
          { \@@_next_page: \space (first~  }
        column)
      }
      { \@@_next_page: }
  }
  { Check~ out~ the~ page~ and~ see~ if~ you~ can~ avoid~ the~ widow.}
%    \end{macrocode}
%
%
%   \bigskip
%
%
%    The case of both widow and orphan is similar, but we obviously
%    need different text so we made it its own message.
%    \begin{macrocode}
\msg_new:nnnn {widows-and-orphans} {orphan-widow}
  { Orphan~
    \legacy_switch_if:nTF {@twocolumn}
       { \legacy_switch_if:nTF {@firstcolumn}
           { and~ #1 widow~ on~ page~ \@@_this_page: \space
             (first~ and~ second~ }
           { on~ page~ \@@_this_page: \space (second~ column)~
             and~ #1 widow~ on~ page~ \@@_next_page: \space (first~ }
       }
       { on~ page~ \@@_this_page: \space (second~ column)~
         and~ #1 widow~ on~ page~ \@@_next_page: \space (first~ }
     column)
  }
  { Check~ out~ the~ page~ and~ see~ if~ you~ can~ avoid~ both~
    orphan~ and~ widow.}
%    \end{macrocode}
%
%
%
%
%
%
% \subsection{Adjusting the parameter values}
%
% To avoid (a lot of) false positives during checking it is important
% that the parameter values are chosen in a way that all possible
% combinations lead to unique \cs{outputenalty} values. At the same
% time, we want them to be as close as possible to the values that
% have been initially requested by the user (or in the document class)
% and if we deviate too much then this will likely alter the way
% \TeX{} does the paginiation.
%  So here is an outline of what we do with the parameters:
% \begin{itemize}
% \item We set up a property list that will hold penalty values that
%    can appear in \cs{outputpenalty} inside the output routine. The
%    penalties are the ``keys'' and the corresponding property list value is
%    the source how they got produced. For example, the key might be
%    \texttt{150} and the value \cs{widowpenalty}.
%
% \item Initially the property list is empty. So adding the first item
%    simply means taking the value of one parameter, say \texttt{150}
%    from $\cs{widowpenalty}+\cs{interlinepenalty}$, as the key and
%    this formula as the property list value.
%
% \item For the next parameter, say \cs{clubpenalty}, we check if its
%    value (or more precisely its value plus \cs{interlinepenalty}) is
%    already a key in the property list. If that is the case, then we
%    failed and have to modify the parameter value somehow.

% \item If not, we also have to check any combination of the current
%    parameter with any parameter processed earlier. If that
%    combination is possible, e.g., \cs{clubpenalty} (new) and
%    \cs{widowpenalty} (already processed) then we also have to check
%    the sum. If that sum is already a key in the property list then
%    we failed as well.
%
% \item If we failed we iterate by incrementing the current parameter
%   value and try again. Eventually we will get to a value where all
%   combinations we test work, that is are not yet in the property list.
%
% \item We then change the parameter to this value and add all the
%    combinations we tried before to the property list (that is
%    $\cs{clubpenalty}+\cs{interlinepenalty}$ alone and together with
%    \cs{widowpenalty} in our example). Thus from now on those are
%    also forbidden values.
%
% \item We do all this with a helper command that takes the new
%    parameter as the first argument and the list of different cases
%    to try as a comma-separated list as a second argument, e.g.,
%\begin{verbatim}
%    \__fmwao_decide_penalty:Nn \clubpenalty 
%             { \clubpenalty + \interlinepenalty ,
%               \clubpenalty + \widowpenalty + \interlinepenalty }
%\end{verbatim}
%
% \item This way we are adding all relevant parameters to the property
%   list and at the same time adjusting their values if needed.
% \item Once all parameters are handled the property list is no longer
%   needed as the parameters got changed along the way (but we keep it
%   around as it allows for a simple display of all settings in one go).
% \end{itemize}
%
%

%  \begin{macro}{\l_@@_penalties_prop}
%    Here is the property list for our process.
%    \begin{macrocode}
\prop_new:N \l_@@_penalties_prop
%    \end{macrocode}
%  \end{macro}
%
%
%  \begin{macro}{\@@_initialize:}
%    Now we are ready to go. First action is to clear the property
%   list as the initialization may happen several times.
%    \begin{macrocode}
\cs_new:Npn \@@_initialize: {
  \prop_clear:N  \l_@@_penalties_prop
%    \end{macrocode}
%    When \TeX{} breaks a page at a glue without an explicit penalty
%    it does set \cs{outputpenalty} to \texttt{10000} in the output
%    routine to distinguish from a case where an explicit penalty of
%    \texttt{0} was in the document. That means none of our parameters
%    or parameter combinations is allowed to have that particular
%    value, because otherwise we would get for each break at a glue a
%    fake match and report an issue. So we enter that value first (by
%    hand) so that it will not be used by a parameter or parameter
%    combination.
%    \begin{macrocode}
  \prop_put:Nnn \l_@@_penalties_prop {10000} {break~ at~ glue}
%    \end{macrocode}
%    The next thing is to add the values for \cs{@lowpenalty},
%    \cs{@medpenalty} and \cs{@highpenalty} to the property list as
%    they also may show up in \cs{outputpenalty} if a user says, for
%    example, \verb=\nopagebreak[2]=.
%
%    Such a penalty from an explicit page break request does not get
%    \cs{interlinepenalty} added on top.
%    \begin{macrocode}
  \@@_decide_penalty:Nn \@lowpenalty  { \@lowpenalty}
  \@@_decide_penalty:Nn \@medpenalty  { \@medpenalty}
  \@@_decide_penalty:Nn \@highpenalty { \@highpenalty}
%    \end{macrocode}
%    Then comes the first real parameter for the orphans:
%    \begin{macrocode}
  \@@_decide_penalty:Nn \clubpenalty { \clubpenalty + \interlinepenalty }
%    \end{macrocode}
%    followed by the one for the widows and the one for the display widows:
%    \begin{macrocode}
  \@@_decide_penalty:Nn \widowpenalty 
    { \widowpenalty + \interlinepenalty ,
      \widowpenalty + \clubpenalty  + \interlinepenalty }
  \@@_decide_penalty:Nn \displaywidowpenalty 
    { \displaywidowpenalty + \interlinepenalty ,
      \displaywidowpenalty + \clubpenalty + \interlinepenalty }
%    \end{macrocode}
%    \cs{brokenpenalty} can appear by it own but also with each and
%    every combination we have seen so far:
%    \begin{macrocode}
  \@@_decide_penalty:Nn \brokenpenalty
    { \brokenpenalty + \interlinepenalty ,
      \brokenpenalty + \clubpenalty + \interlinepenalty ,
      \brokenpenalty + \widowpenalty + \interlinepenalty , 
      \brokenpenalty + \widowpenalty + \clubpenalty + \interlinepenalty ,
      \brokenpenalty + \displaywidowpenalty + \clubpenalty + \interlinepenalty }
%    \end{macrocode}
%    Finally there is the one for lonely displays (again without
%    \cs{interlinepenalty} being added):
%    \begin{macrocode}
  \@@_decide_penalty:Nn \predisplaypenalty { \predisplaypenalty }
}
%    \end{macrocode}
%    If we run the above code with the default parameter settings of \LaTeX{}
%    in force it will make a few adjustments and the property list
%    would afterwards contain
%    the following entries:
%\begin{small}
%\begin{verbatim}
% The property list \l__fmwao_penalties_prop contains the pairs (without
% outer braces):
% >  {10000}  =>  {break at glue}
% >  {51}  =>  {\@lowpenalty }
% >  {151}  =>  {\@medpenalty }
% >  {301}  =>  {\@highpenalty }
% >  {150}  =>  {\clubpenalty +\interlinepenalty }
% >  {152}  =>  {\widowpenalty +\interlinepenalty }
% >  {302}  =>  {\widowpenalty +\clubpenalty +\interlinepenalty }
% >  {50}  =>  {\displaywidowpenalty +\interlinepenalty }
% >  {200}  =>  {\displaywidowpenalty +\clubpenalty +\interlinepenalty }
% >  {100}  =>  {\brokenpenalty +\interlinepenalty }
% >  {250}  =>  {\brokenpenalty +\clubpenalty +\interlinepenalty }
% >  {252}  =>  {\brokenpenalty +\widowpenalty +\interlinepenalty }
% >  {402}  =>  {\brokenpenalty +\widowpenalty +\clubpenalty +\interlinepenalty }
% >  {300}  =>  {\brokenpenalty +\displaywidowpenalty +\clubpenalty +\interlinepenalty }
% >  {10001}  =>  {\predisplaypenalty }.
%\end{verbatim}
%\end{small}
%  \end{macro}



%  \begin{macro}{\l_@@_tmp_int}
%  \begin{macro}{\l_@@_success_bool}
%    For doing the actual calculations and insertions into the
%    property list we also need an integer register, a token list
%    variable and another boolean variable.
%    \begin{macrocode}
\int_new:N  \l_@@_tmp_int
\tl_new:N   \l_@@_tmp_tl
\bool_new:N \l_@@_success_bool
%    \end{macrocode}
%  \end{macro}
%  \end{macro}
%
%
%  \begin{macro}{\@@_decide_penalty:Nn}
%    This is the core command doing the real work. Remember its first
%    argument is the parameter which we are currently handling and the
%    second argument is a comma-separated list of cases for which we
%    need make sure that their result is not yet in the property list.
%    \begin{macrocode}
\cs_new:Npn \@@_decide_penalty:Nn #1 #2 {
%    \end{macrocode}
%    We start by setting the boolean to \texttt{false} and then run a
%    loop until we have found a suitable parameter value that meets
%    our criteria.
%    \begin{macrocode}
  \bool_set_false:N \l_@@_success_bool
  \bool_do_until:Nn \l_@@_success_bool
    {
%    \end{macrocode}
%    Inside the loop we start with the assumption that the current
%    value of the parameter is fine and then check if that assumption
%    is true. If yes, we can exit the loop otherwise we will have to
%    try with a different value.
%    \begin{macrocode}
     \bool_set_true:N \l_@@_success_bool
%    \end{macrocode}
%    For the verification we try each item in the second parameter to
%    see if that is already in the property list.
%    This means evaluating the expression to get the penalty value
%    and then looking it up in the property list. If it is there, we
%    failed. So in this case we set the boolean back to false and
%    break out of the mapping over the second argument since there is
%    no point in testing further.
%    \begin{macrocode}
     \clist_map_inline:nn { #2 }
       { \int_set:Nn \l_@@_tmp_int {##1}
         \prop_get:NVNT
	    \l_@@_penalties_prop \l_@@_tmp_int \l_@@_tmp_tl
            { \clist_map_break:n{\bool_set_false:N\l_@@_success_bool} }
       }
%    \end{macrocode}
%    Once we have finished, the boolean will tell us if we are
%    successful so far. If yes, it means there was no conflict.  We
%    can therefore add all combinations with this parameter to the
%    property list, as from now on they are forbidden as well. So we
%    map once more over the second argument and enter them:
%    \begin{macrocode}
     \bool_if:NTF \l_@@_success_bool
       { \clist_map_inline:nn { #2 }
            { \int_set:Nn \l_@@_tmp_int {##1}
              \prop_put:NVn \l_@@_penalties_prop \l_@@_tmp_int { ##1 } }
       }
%    \end{macrocode}
%    If we failed we increment the parameter value and retry:
%    \begin{macrocode}
       { \int_incr:N #1 }
     }
}
%    \end{macrocode}
%
%
%    One place where we will run this code is at the beginning of the
%    document (so that changes to the parameters in the preamble or
%    the document class are picked up). The other place is when the
%    user changes any of the parameters in the middle of the document
%    via \cs{WaOsetup}.
%    \begin{macrocode}
\AtBeginDocument { \@@_initialize: }
%    \end{macrocode}
%
%  \end{macro}
%
% \subsection{The option setup}
%
% The options are fairly straight forward:
%
%    \begin{macrocode}
\keys_define:nn {fmwao} {
%    \end{macrocode}
%    By default messages are given as warnings above. If anything else
%    is wanted the option \option{check} can be used which simply
%    changes the message class internally:
%    \begin{macrocode}
  ,check .choice:
  ,check / error
     .code:n = \msg_redirect_module:nnn {widows-and-orphans}{warning}{error}
  ,check / info
     .code:n = \msg_redirect_module:nnn {widows-and-orphans}{warning}{info}
  ,check / none
     .code:n = \msg_redirect_module:nnn {widows-and-orphans}{warning}{none}
  ,check / warning
     .code:n = \msg_redirect_module:nnn {widows-and-orphans}{warning}{ }
%    \end{macrocode}
%    The other options set parameters to some hopefully ``reasonable''
%   values---no real surprises here. \LaTeX{} internally uses
%   \cs{@clubpenalty} so we need to set this too, if we change
%   \cs{clubpenalty}.
%    \begin{macrocode}
   % ====================================
  ,orphans .choice:
  ,orphans / prevent .code:n = \int_set:Nn \clubpenalty  { 10000        }
                               \int_set:Nn \@clubpenalty { \clubpenalty }
  ,orphans / avoid   .code:n = \int_set:Nn \clubpenalty  { 5000         }
                               \int_set:Nn \@clubpenalty { \clubpenalty }
  ,orphans / default .code:n = \int_set:Nn \clubpenalty  { 150          }
                               \int_set:Nn \@clubpenalty { \clubpenalty }
   % ====================================
  ,widows .choice:
  ,widows / prevent  .code:n = \int_set:Nn \widowpenalty { 10000 }
  ,widows / avoid    .code:n = \int_set:Nn \widowpenalty { 5000  }
  ,widows / default  .code:n = \int_set:Nn \widowpenalty { 150   }
   % ====================================
  ,hyphens .choice:
  ,hyphens / prevent .code:n = \int_set:Nn \brokenpenalty { 10000 }
  ,hyphens / avoid   .code:n = \int_set:Nn \brokenpenalty { 2000  }
  ,hyphens / default .code:n = \int_set:Nn \brokenpenalty { 50    }
   % ====================================
  ,prevent-all  .code:n = \int_set:Nn \clubpenalty        { 10000   }
                          \int_set:Nn \widowpenalty       { 10000   }
                          \int_set:Nn \displaywidowpenalty{ 10000   }
                          \int_set:Nn \brokenpenalty      { 10000   }
                          \int_set:Nn \predisplaypenalty  { 10000   }
                          \int_set:Nn \@clubpenalty  { \clubpenalty }
%    \end{macrocode}
%    As an exception option \option{avoid-all} doesn't set
%    \cs{predisplaypenalty}, maybe it should.
%    \begin{macrocode}
  ,avoid-all    .code:n = \int_set:Nn \clubpenalty  {        5000   }
                          \int_set:Nn \widowpenalty {        5000   }
                          \int_set:Nn \displaywidowpenalty { 2000   }
                          \int_set:Nn \brokenpenalty{        2000   }
%                         \int_set:Nn \predisplaypenalty{    9999   }
                          \int_set:Nn \@clubpenalty  { \clubpenalty }
%    \end{macrocode}
%    \option{default-all} reverts back to the standard \LaTeX{}
%    default values:
%    \begin{macrocode}
  ,default-all  .code:n = \int_set:Nn \clubpenalty  {         150   }
                          \int_set:Nn \widowpenalty {         150   }
                          \int_set:Nn \displaywidowpenalty {   50   }
                          \int_set:Nn \brokenpenalty{         100   }
                          \int_set:Nn \predisplaypenalty{   10000   }
                          \int_set:Nn \@clubpenalty  { \clubpenalty }
}
%    \end{macrocode}
%    Once declared we evaluate the options given to the package:
%    \begin{macrocode}
\ProcessKeysPackageOptions{fmwao}
%    \end{macrocode}
%
%
% \subsection{Document-level commands}
%
% Finally we declare three user-level commands:
%

%  \begin{macro}{\WaOsetup}
%    This runs the key setup on the first argument and the
%    reinitializes the parameter setup accordingly.
%    \begin{macrocode}
\NewDocumentCommand\WaOsetup{m}
  { \keys_set:nn{fmwao}{#1}  \@@_initialize: \ignorespaces }
%    \end{macrocode}
%  \end{macro}
%
%

%  \begin{macro}{\WaOparameters}
%    With this parameterless command one can get a display of the
%    current parameter settings.
%    \begin{macrocode}
\NewDocumentCommand\WaOparameters{}{\prop_show:N \l_@@_penalties_prop}
%    \end{macrocode}
%  \end{macro}
%
%
%  \begin{macro}{\WaOignorenext}
%    And here is the command that suppresses any warning on the
%    current page or column:
%    \begin{macrocode}
\NewDocumentCommand\WaOignorenext{}{\bool_gset_false:N\g_@@_gen_warn_bool}
%    \end{macrocode}
%  \end{macro}
%
% \Finale
%

\endinput
